\chapter{Swapping order with Lebesgue integrals}
\label{ch:swapsum_measure}
\section{Motivating limit interchange}
\prototype{$\mathbf{1}_\QQ$ is good!}

One of the issues with the Riemann integral is
that it behaves badly with respect to convergence of functions,
and the Lebesgue integral deals with this.
This is therefore often given as a poster child
on why the Lebesgue integral has better behaviors than the Riemann one.

We technically have already seen this:
consider the indicator function $\mathbf{1}_\QQ$,
which is not Riemann integrable by \Cref{prob:1QQ}.
But we can readily compute its Lebesgue integral over $[0,1]$, as
\[ \int_{[0,1]} \mathbf{1}_\QQ \; d\mu
	= \mu\left( [0,1] \cap \QQ \right) = 0 \]
since it is countable.

This \emph{could} be thought of as a failure of existence
for the Riemann integral.
\begin{example}
	[$\mathbf{1}_\QQ$ is a limit of finitely supported functions]
	\label{ex:1QQindicator}
	We can define the sequence of functions $g_1$, $g_2$, \dots\ by
	\[ g_n(x) = \begin{cases}
			1 & (n!)x \text{ is an integer} \\
			0 & \text{else}.
		\end{cases} \]
	Then each $g_n$ is piecewise continuous
	and hence Riemann integrable on $[0,1]$ (with integral zero),
	but $\lim_{n \to \infty} g_n = \mathbf{1}_\QQ$ is not.
\end{example}

The limit here is defined in the following sense:
\begin{definition}
	Let $f$ and $f_1, f_2, \dots \colon \Omega \to \RR$ be a sequence of functions.
	Suppose that for each $\omega \in \Omega$, the sequence
	\[ f_1(\omega), \; f_2(\omega), \; f_3(\omega), \;, \dots \]
	converges to $f(\omega)$.
	Then we say $f_1$, $f_2$, \dots\ \vocab{converges pointwise}
	to the limit $f$, written $\lim_{n \to \infty} f_n = f$.

	We can define $\liminf_{n \to \infty} f_n$
	and $\limsup_{n \to \infty} f_n$ similarly.
\end{definition}

By ``the Lebesgue integral has better behavior'', we means the following:
\begin{proposition}
	If $f_1, f_2, \dots \colon \Omega \to \RR$ are measurable functions,
	then $\liminf_{n \to \infty} f_n$ and $\limsup_{n \to \infty} f_n$ are measurable.
\end{proposition}
When $f_n$ are all nonnegative, this means
$\int_\Omega \liminf_{n \to \infty} f_n d\mu$ and $\int_\Omega \limsup_{n \to \infty} f_n d\mu$ exists.
(If they can be negative, the behavior is not that nice. \Cref{prob:sin_improper} gives an example.)

Unfortunately, even if the integral exists, we can't always exchange pointwise limit with Lebesgue
integral.

Why would we want to? For instance, if we face this problem:
\begin{quote}
	Compute $\lim_{k \to \infty} \int_1^\infty \frac{1}{k} e^{-x^2} \; dx$.
\end{quote}
While the integral $\int e^{-x^2} dx$ is not computable by elementary means,
we would like to say the limit is simply $0$ (why wouldn't it be?)

Unfortunately, pointwise convergence is actually a fairly weak notion of convergence.

\begin{example}
	\label{ex:failure_interchange_lim_int}
	In all of these examples, we cannot interchange the limit and the integral
	without changing the result.
	\begin{itemize}
		\ii The sequence $f_k(x) = \frac{\sin(x)}{x} \cdot \mathbf{1}_{(1, k)}$ converges pointwise to
		$f(x) = \frac{\sin(x)}{x} \cdot \mathbf{1}_{(1, \infty)}$ as $k \to \infty$,
		and the limit $\lim_{k \to \infty} \int f_k(x) \; dx$ exists, but $f$ is not integrable.
		\ii Similarly, $f_k(x) = \frac{\sin(1/x)}{x} \cdot \mathbf{1}_{(1/k, \infty)}$
		converges pointwise to $f(x) = \frac{\sin(1/x)}{x} \cdot \mathbf{1}_{(0, \infty)}$
		as $k \to \infty$,
		the limit $\lim_{k \to \infty} \int f_k(x) \; dx$ exists and is finite,
		but $f$ is not integrable.
		\ii The sequence $f_k(x) = \frac{\mathbf{1}_{(0, k)}}{k}$ converges pointwise to $f(x)=0$ as $k
		\to \infty$, for every $k$ then $\int f_k(x) \; dx=1$, but $\int f(x) \; dx=0$.

		Note that, in this case, the convergence is actually uniform!
		\ii We don't even need $k$ in the denominator --- the sequence $f_k(x) = \mathbf{1}_{(0, k)}$ also
		converges pointwise to $f(x)=0$, but this time, for every $k$ then $\int f_k(x) \; dx=\infty$!
		\ii The sequence $f_k(x) = k \cdot \mathbf{1}_{(0, 1/k)}$ converges pointwise to $f(x)=0$ as $k
		\to \infty$. But similar to above, $\int f_k(x) \; dx=1$ for every $k$, but
		$\int f(x) \; dx=0$.
	\end{itemize}
	The last example is similar in behavior to an example known as the Witch's hat.\footnote{%
	\url{https://www.geogebra.org/m/dv7ctmed} has an animation.}
\end{example}

As such, the convergence theorems stated below is an attempt to classify all the possible anomalies,
and to show that in ``usual'' cases, interchanging limit and integral just works.

As mentioned earlier, we choose to use the Lebesgue integral instead of the Riemann integral,
because in such cases, the Lebesgue integral will usually just exist.

%This is why when thinking about the Riemann integral
%it is commonplace to work with stronger conditions like
%``uniformly convergent'' and the like.
%However, with the Lebesgue integral, we can mostly not think about these!

\section{Overview}
The three big-name results for exchanging
pointwise limits with Lebesgue integrals is:
\begin{itemize}
	\ii Fatou's lemma: the most general statement possible,
	for any nonnegative measurable functions.
	\ii Monotone convergence: ``increasing limits'' just work.
	\ii Dominated convergence (actually Fatou-Lebesgue):
	limits that are not too big
	(bounded by some absolutely integrable function) just work.
\end{itemize}

\section{Fatou's lemma}

In all the above examples, we see that:
\begin{itemize}
	\ii The failure of the interchange of limit and integral is caused by
	the functions in the sequence have too much room to ``wiggle around'', and
	\ii as such, the integrals $\int f_k(x) dx$ are all greater than the integral of the limit $\int f(x)
	dx$.
\end{itemize}
Of course, by negating all the functions $f_k(x)$ we can get $\lim_{k \to \infty} \int f_k(x) dx < \int
f(x) dx$.
But, as it turns out, for nonnegative functions, \emph{this sort of behavior is the only behavior
possible}. In other words,
\begin{moral}
	For nonnegative functions, if limit of integral is not equal to integral of limit, the former one is
	always larger.
\end{moral}

\begin{lemma}
	[Fatou's lemma, preliminary version]
	Let $f_1, f_2, \dots \colon \Omega \to [0,+\infty]$
	be a sequence of \emph{nonnegative} measurable functions, converging pointwise to $f$.
	Then $f$ is nonnegative, measurable, and
	\[ \int_\Omega f \; d\mu
		\le \lim_{n \to \infty} \left( \int_\Omega f_n \; d\mu \right).  \]
	Here we allow either side to be $+\infty$.
\end{lemma}

As it turns out, this lemma can significantly be generalized as follows.
If you compare the two statements, you can see the two $\lim$ operators are changed to $\liminf$ ---
when the sequence actually converges, $\liminf$ and $\lim$ equals.
\begin{lemma}
	[Fatou's lemma]
	Let $f_1, f_2, \dots \colon \Omega \to [0,+\infty]$
	be a sequence of \emph{nonnegative} measurable functions.
	Then $\liminf_{n \to \infty} f_n \colon \Omega \to [0,+\infty]$ is measurable and
	\[ \int_\Omega \left( \liminf_{n \to \infty} f_n \right) \; d\mu
		\le \liminf_{n \to \infty} \left( \int_\Omega f_n \; d\mu \right).  \]
	Here we allow either side to be $+\infty$.
\end{lemma}
Notice that there are \emph{no extra hypothesis}
on $f_n$ other than nonnegative: which makes this quite surprisingly versatile
if you ever are trying to prove some general result.

\section{Everything else}
The big surprise is how quickly all the ``big-name''
theorem follows from Fatou's lemma.
Here is the so-called ``monotone convergence theorem''.
\begin{corollary}
	[Monotone convergence theorem]
	Let $f$ and $f_1, f_2, \dots \colon \Omega \to [0,+\infty]$
	be a sequence of \emph{nonnegative}
	measurable functions such that $\lim_n f_n = f$
	and $f_n(\omega) \le f(\omega)$ for each $n$.
	Then $f$ is measurable and
	\[ \lim_{n \to \infty} \left( \int_\Omega f_n \; d\mu \right)
		= \int_\Omega f \; d\mu. \]
	Here we allow either side to be $+\infty$.
\end{corollary}
\begin{proof}
	We have
	\begin{align*}
		\int_\Omega f \; d\mu
		&= \int_\Omega \left( \liminf_{n \to \infty} f_n \right) \; d\mu \\
		&\le \liminf_{n \to \infty} \int_\Omega f_n \; d\mu \\
		&\le \limsup_{n \to \infty} \int_\Omega f_n \; d\mu \\
		&\le \int_\Omega f \; d\mu
	\end{align*}
	where the first $\le$ is by Fatou lemma,
	and the third by the fact that
	$\int_\Omega f_n \le \int_\Omega f$ for every $n$.
	This implies all the inequalities are equalities and we are done.
\end{proof}
You can see how short the proof is ---
proving $\limsup_{n \to \infty} \int_\Omega f_n \; d\mu \le \int_\Omega f \; d\mu$ is the easy half,
and the difficult half is automatically taken care of by Fatou's lemma.

\begin{remark}
	[The monotone convergence theorem does not require monotonicity!]
	In the literature it is much more common
	to see the hypothesis $f_1(\omega) \le f_2(\omega) \le \dots \le f(\omega)$
	rather than just $f_n(\omega) \le f(\omega)$ for all $n$,
	which is where the theorem gets its name.
	However as we have shown this hypothesis is superfluous!
	This is pointed out in \url{https://mathoverflow.net/a/296540/70654},
	as a response to a question entitled
	``Do you know of any very important theorems that remain unknown?''.
\end{remark}

\begin{example}
	[Monotone convergence gives $\mathbf{1}_\QQ$]
	This already implies \Cref{ex:1QQindicator}.
	Letting $g_n$ be the indicator function for $\frac1{n!}\ZZ$
	as described in that example, we have $g_n \le \mathbf{1}_\QQ$
	and $\lim_{n \to \infty} g_n(x) = \mathbf{1}_\QQ(x)$,
	for each individual $x$.
	So since $\int_{[0,1]} g_n \; d\mu = 0$ for each $n$,
	this gives $\int_{[0,1]} \mathbf{1}_\QQ = 0$ as we already knew.
\end{example}

The most famous result, though is the following.
\begin{corollary}
	[Fatou-Lebesgue theorem]
	Let $f$ and $f_1, f_2, \dots \colon \Omega \to \RR$
	be a sequence of measurable functions.
	Assume that $g \colon \Omega \to \RR$ is an
	\emph{absolutely integrable} function for which
	$|f_n(\omega)| \le |g(\omega)|$ for all $\omega \in \Omega$.
	Then the inequality
	\begin{align*}
		\int_\Omega \left( \liminf_{n \to \infty} f_n \right) \; d\mu
		&\le \liminf_{n \to \infty} \left( \int_\Omega f_n \; d\mu \right) \\
		&\le \limsup_{n \to \infty} \left( \int_\Omega f_n \; d\mu \right)
		\le \int_\Omega \left( \limsup_{n \to \infty} f_n \right) \; d\mu.
	\end{align*}
\end{corollary}
\begin{proof}
	There are three inequalities:
	\begin{itemize}
		\ii The first inequality follows by Fatou on $g + f_n$ which is nonnegative.
		\ii The second inequality is just $\liminf \le \limsup$.
		(This makes the theorem statement easy to remember!)
		\ii The third inequality follows by Fatou on $g - f_n$ which is nonnegative.
		\qedhere
	\end{itemize}
\end{proof}

\begin{exercise}
	Where is the fact that $g$ is absolutely integrable used in this proof?
\end{exercise}

\begin{corollary}
	[Dominated convergence theorem]
	Let $f_1, f_2, \dots \colon \Omega \to \RR$
	be a sequence of measurable functions
	such that $f = \lim_{n \to \infty} f_n$ exists.
	Assume that $g \colon \Omega \to \RR$ is an
	\emph{absolutely integrable} function for which
	$|f_n(\omega)| \le |g(\omega)|$ for all $\omega \in \Omega$.
	Then
	\[ \int_\Omega f \; d \mu
		= \lim_{n \to \infty} \left( \int_\Omega f_n \; d\mu \right). \]
\end{corollary}

In other words,
\begin{moral}
	If there's only finite ``space'' for the functions $f_k$ to ``wiggle around'',
	then no anomaly can happen.
\end{moral}

\begin{proof}
	If $f(\omega) = \lim_{n \to \infty} f_n(\omega)$,
	then $f(\omega) = \liminf_{n \to \infty} f_n(\omega)
	= \limsup_{n \to \infty} f_n(\omega)$.
	So all the inequalities in the Fatou-Lebesgue theorem
	become equalities, since the leftmost and rightmost sides are equal.
\end{proof}
Note this gives yet another way to verify \Cref{ex:1QQindicator}.
In general, the dominated convergence theorem
is a favorite clich\'{e} for undergraduate exams,
because it is easy to create questions for it.
Here is one example showing how they all look.
\begin{example}
	[The usual Lebesgue dominated convergence examples]
	Suppose one wishes to compute
	\[ \lim_{n \to \infty}
		\left( \int_{(0,1)} \frac{n\sin(n\inv x)}{\sqrt x} \right) \; dx \]
	then one starts by observing that
	the inner term is bounded by the absolutely integrable function $x^{-1/2}$.
	Therefore it equals
	\begin{align*}
		\int_{(0,1)} \lim_{n \to \infty}
			\left( \frac{n\sin(n\inv x)}{\sqrt x} \right) \; dx
		&= \int_{(0,1)} \frac{x}{\sqrt x} \; dx \\
		&= \int_{(0,1)} \sqrt{x} \; dx = \frac23.
	\end{align*}
\end{example}

We can also say something else about the behavior of the anomalies --- that is, when
$|\Omega| < \infty$, the anomaly only happens in a set of \emph{small measure}.

\begin{theorem}
	[Egorov's theorem]
	Let $f_1, f_2, \dots \colon \Omega \to \RR$
	be a sequence of measurable functions, on a measure space $\Omega$ with $|\Omega| < \infty$,
	such that $f = \lim_{n \to \infty} f_n$ exists and is finite almost everywhere.
	Then, for any $\eps>0$,
	we can find a subset $U \subseteq \Omega$, such that
	the remainder has small measure:
	\[
		|\Omega \smallsetminus U| < \eps,
	\]
	and the convergence is uniform on $U$: the sequence
	\[
		f_1|_U, f_2|_U, \dots
	\]
	converges to $f_U$ uniformly.
\end{theorem}

This is because of the following theorem.
\begin{theorem}
	[Uniform convergence theorem]
	Let $f_1, f_2, \dots \colon \Omega \to \RR$
	be a sequence of integrable functions, on a measure space $\Omega$ with $|\Omega| < \infty$,
	such that $\lim_{n \to \infty} f_n = f$, and the convergence is uniform.
	Then $f$ is integrable and,
	\[ \lim_{n \to \infty} \left( \int_\Omega f_n \; d\mu \right)
		= \int_\Omega f \; d\mu. \]
\end{theorem}

In other words,
\begin{moral}
	The fact that $\int f \; d\mu \neq \lim \int f_k \; d\mu$ is only caused by
	$\int_{ \Omega \setminus U } f \; d\mu \neq \lim \int_{ \Omega \setminus U } f \; d\mu$.
\end{moral}

\begin{example}
	[Removing a set of small measure will allow interchanging the integral and the limit]
	We take a few examples from \Cref{ex:failure_interchange_lim_int}, and see what happens if we remove
	a set of small measure here.
	\begin{itemize}
		\ii Consider the sequence $f_k(x) = k \cdot \mathbf{1}_{(0, 1/k)}$. If, for any $\eps>0$,
		we delete a segment $(0, \eps)$ from the domain of $f_k$, then we will have that $f_k$ converges
		uniformly to $f$ as $k \to \infty$, and that $\lim_{k \to \infty} \int f_k(x) \; dx = \int f(x)
		\; dx = 0$.
		\ii Similarly, the sequence $f_k(x) = \frac{\sin(1/x)}{x} \cdot \mathbf{1}_{(1/k, 1)}$
		converges pointwise to $f(x) = \frac{\sin(1/x)}{x} \cdot \mathbf{1}_{(0, 1)}$,
		and if we delete a segment $(0, \eps)$, then everything checks out.
	\end{itemize}
\end{example}

\begin{remark}
	Just because we only need to delete a set of small measure, doesn't mean the set is concentrated in a
	small interval. The reader is invited to construct a sequence $f_k \colon [0, 1] \to \RR^+$
	that converges pointwise to $f$, but in order to make the convergence uniform, a dense subset of $[0,
	1]$ need to be removed.
	(Hint: take any discontinuous everywhere nonnegative function $f$, and set $f_k = \min(k, f)$.)
\end{remark}

\section{Fubini and Tonelli}
\todo{TO BE WRITTEN}

\section{\problemhead}
\todo{problems}
