\chapter{Bonus: Topological Abel-Ruffini Theorem}
\label{ch:abel_ruffini_theorem}
We've already shown the Fundamental Theorem of Algebra.
Now, with our earlier intuition on holomorphic $n$th roots,
we can now show that there is no general formula for
the roots of a quintic polynomial.

\section{The Game Plan}
Firstly, what do we even mean by ``formula'' here?

\begin{definition}
	A \vocab{quintic formula} would be a formula taking in the coefficients
	$(a_0, \dots, a_5)$ of a degree $5$ polynomial $P$,
	using the operations $+$, $-$, $\times$, $\div$, $\sqrt[n]{}$
	finitely many times, that maps to the five roots $(z_1, \cdots, z_5)$ of $P$.
\end{definition}

Now, any proposed quintic formula $F$ receives the same coefficients when the roots are the same,
and thus gives the same output. This is fine at first glance, but swapping two roots continuously
might pose more issues. $F$ must create and preserve some order of the roots under these permutations.

\begin{ques}
	Convince yourself any $F$ indeed must track which root is which when moving roots along
	smooth paths.
\end{ques}

\begin{remark}
	This isn't true if we bring even more complicated functions
	such as \vocab{Bring Radicals} to the table.
	But this wasn't really considered ``fair game.''
\end{remark}

\section{Step 1: The Simplest Case}
Let's first ignore the $\sqrt[n]{}$ operator for motivation.
Suppose I told you that some rational function $R$ always finds
a root of a quintic polynomial $P(z) = (z - z_1)(z - z_2)(z - z_3)(z - z_4)(z - z_5)$.
For simplicity, let all the roots be distinct.

Suppose that initially $R$ outputs $z_1$. Consider what happens we smoothly swap the roots $z_1$ and $z_2$
along two non-intersecting paths that doesn't go through other roots.

\begin{center}
	%% First Diagram
	\begin{asy}
		size(7cm);
		cplane();
		pair T;
		pair[] A = new pair[5];

		real r = 1.44;
		real[] dirs = {25,120,155,215,325};
		real[] mag = {1.5,1.9,1.25,0.9,1.8};
		for (int i=0; i<5; ++i) {
			A[i] = mag[i]*dir(dirs[i]);
			if (i < 2) {
				dot("$z_{" + (string) (i+1) + "}$", A[i], dir(dirs[i]), red);
			}
			else {
				dot("$z_{" + (string) (i+1) + "}$", A[i], dir(dirs[i]), blue);
			}
		}
		draw(A[0]..(0.8*A[0]+0.6*A[1])..A[1], red, EndArrow);
		draw(A[1]..(0.2*A[0]+0.4*A[1])..A[0], red, EndArrow);
	\end{asy}
\end{center}

Since $R$ is continuous, it must be tracking the same root. However, once we finish
swapping $z_1$ and $z_2$, the coefficients of $P$ are the same as they were initially.
But this means that $R$ has been tricked into changing the root it outputs, contradiction!

The bigger picture here is that we were able to find an operation that fixes $R$ while changing the order
of the roots in $S_5$.

\section{Step 2: Nested Roots}
Once we add $\sqrt[n]{}$ back to the picture, this idea no longer works right out of the box.

\begin{example}[Quadratic Formula]
	If you've done any competition math, you know that for a polynomial
	$P(z) = az^2 + bz + c = (z-z_1)(z - z_2)$, it follows that the two branches of
	\[
		\frac{-b \pm \sqrt{b^2 - 4ac}}{2a}
	\]
	give $z_1$ and $z_2$.

	So why can't swapping $z_1$ and $z_2$ yield a contradiction here?
	It's because while all the coefficients end up in the starting position,
	the liftings of how $\sqrt{b^2 - 4ac}$ travels may not.
\end{example}

\begin{exercise}
	Consider the polynomial $z^2 - 1$. Then smoothly swap the roots to get the intermediary
	polynomials of $(z - e^{it})(z + e^{it})$. See that the two roots given by the quadratic formula
	also swap position.
\end{exercise}

Let's now consider the next simplest case of the $n$th root of a rational function
$\sqrt[n]{R}$, and try to fix it with a nontrivial permutation of the roots.

Swapping the roots $z_1$ and $z_2$, we keep $R$ the same, but $R$'s path $\alpha$ around
the origin may have accumulated some change in phase $2\pi a$. If we were to unswap
$z_1$ and $z_2$ in the same manner, we'd undo the change in phase, but we'd also be back
to doing nothing.

However, while changes in phase are \emph{abelian}, permutations are not. Let's consider
another operation of swapping the roots $z_2$ and $z_3$. Taking a commutator of the two
operations, we keep all the phases the same, but end up with a permutation
$(12)(23)(12)^{-1}(23)^{-1}$.

If we mark the second operation's path with $\beta$, this corresponds to
$\alpha\beta\alpha^{-1}\beta^{-1}$.

\begin{center}
	\begin{asy}
		unitsize(0.8cm);
		size(7cm);
		cplane();
		pair A = Drawing((1.5,0));
		dot("$R$", A, dir(-45));
		draw(A..(0,2.5)..MP("\alpha", (-1.5,2), dir(90))..(-1.2,-1.3)..(1,1.5)..(1.8,1.1)..A, red, EndArrow);
		draw(A..(0.7,0.5)..MP("\beta", (0.6,0.3), dir(90))..A, blue, EndArrow);
	\end{asy}
\end{center}

\begin{exercise}
	Show that this permutation operation is nontrivial.
\end{exercise}

We now have better tools:
We have permutations in $S_5$ that fix the $n$th roots of rational functions,
and their compositions under $+$, $-$, $\times$, $\div$.

How do we handle the nested radicals now?

\begin{example}[Cubic Formula]
	The cubic formula contains a nasty term
	\[
		\sqrt[3]{\frac{2b^3 - 9abc + 27a^2d + \sqrt{(2b^3 - 9abc + 27a^d)^2 - 4(b^3 - 3ac)^3}}{2}}.
	\]
	Here, we've taken multiple roots.
\end{example}

\begin{definition}
	Define the degree of a nested radical as the maximum number
	of times radicals can be found in other radicals.
\end{definition}

Let's now consider nested radicals of degree $2$, such as say
$\sqrt[3]{\sqrt{ab + c} - \sqrt{d}}$. We know that we have nontrivial commutators
$\sigma$ and $\rho$ that fix the interior of the cube root, but once again the phase
may not be preserved under each operation individually. Once again,
we can again consider the \emph{commutators} of these commutators,
say $\sigma\rho\sigma^{-1}\rho^{-1}$
which by the same logic fixes the issues with phase.

There's no reason, we can't consider the commutators of commutators of commutators to fix
radicals of degree $3$ and so on. It thus just remains that we always can keep getting
nontrivial commutators.

\section{Step 3: Normal Groups}
We've reduced this to a group theory problem. Given a chain of commutators
\[
	S_5 = G \supseteq G^{(1)} \supseteq G^{(2)} \supseteq \dots
\]
where each group is the commutator subgroup of the next,
we want to show that $G^{(n)}$ never becomes trivial.
This chain is called the \vocab{derived series}.

\begin{exercise}
	Show that for the commutator subgroup $[G,G]$ of a group $G$,
	we have that $[G,G] \normalin G$, and that $G/[G,G]$ is Abelian.
\end{exercise}

\begin{definition}
	A group $G$ is \vocab{solvable} if its derived series is nontrivial.
\end{definition}

So all that remains is showing that $S_5$ is not solvable.
This is a calculation that isn't relevant to the topology ideas in this chapter,
so we defer it to \Cref{prob:A_5_not_solvable}.

\section{Summary}
While this is indeed a valid proof, it has some pros and cons.
As a con, we haven't shown that any polynomial such as $z^5 - z - 1$
has a root that can't be expressed using nested $n$th roots. We've only
that we don't have a formula for all degree $5$ polynomials.

As a pro, this argument makes it easy to add even more functions such
as $\exp$, $\sin$, and $\cos$ to the mix and show even then that no such formula exists.
It also allows you to broadly understand what people mean when they compare
this theorem to a fact that $A_5$ is not solvable.

\section\problemhead

\begin{problem}
	\onechili
	\label{prob:A_5_not_solvable}
	Show that $A_5$ is not solvable.
\end{problem}
